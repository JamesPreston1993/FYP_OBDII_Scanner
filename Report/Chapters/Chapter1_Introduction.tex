\section{Summary}
	\paragraph{}{
	This project is focused on the development of an application capable of communicating with a vehicle and providing feedback on potential issues with it. The information that the application gathers can then be used by the end user when negotiating guesstimates on service charges with professionals, such as mechanics. The intention is that an end user with no technical or mechanical knowledge about their car can perform a self check to get an estimation of the severity of the issues with the vehicle along with the cost associated with fixing them.
	}
	\paragraph{}{
	The project will provide a number of benefits for the target audience: the average vehicle owner. Firstly, it will provide them with a better knowledge of the state of their vehicle. They will be able to use this information to obtain a better knowledge of the inner workings of their vehicle and to determine if the car needs a check up in the near future. It will also save the end user time and money, by allowing them to self check the car rather than paying for a professional to do it for them.
	}
	\paragraph{}
	{
	The architecture of the application will be highly extensible allowing for new functionality to be added with relative ease. It will also allow for the application to be ported to other operating systems, such as Android and iOS. The extensibility of the architecture of the application will be evaluated by porting a subset of the features to Android. This will be accomplished using the Xamarin Framework, which allows for writing Android and iOS applications using C{\#}.
	}
	\paragraph{}
	{
	The application will be evaluated through field testing by connecting to live vehicles. This testing will involve establishing and maintaining communications with the vehicle and gathering the relevant data from it. The data gathered will then be compared to that of similar applications to verify that the application functions correctly.
	}
	
\section{Overview of Domain}
	\paragraph{}{
	Over the years, cars have been transformed from purely mechanical tools to more sophisticated technical machines. Most modern vehicles now contain built in computers, called Engine Control Units (ECU), that monitor and control its various subsystems, for example the Anti-lock Brake System (ABS). As the complexity of vehicles grew, it became harder to track defects. While issues in earlier vehicles could be narrowed down to a mechanical failure, modern vehicles can have mechanical or electrical faults. Now, all modern vehicles have an On-board Diagnostics (OBDII) port, which allows diagnostic tools to connect to the vehicle. The OBDII standard also defines a subset of data that must be retrievable from all compliant vehicles, and the messaging format of the data.		 
	}
	\paragraph{}{
	The application will use an ELM327 Bluetooth dongle to connect to and communicate with the vehicle. This is a micro-controller that connects to the OBDII port found in modern vehicles. The ELM327 abstracts the communication process by including it's own commands that are passed to the vehicle. This removes the need for low level communication between the application and the vehicle. The ELM327 also has a number of benefits for the end user. Firstly, they are inexpensive and widely available, meaning the end user doesn't have to spend time or money sourcing the device from a specific vendor. The device also requires no setup, as the application will handle this automatically. Overall, the ELM327 benefits the developer by having a universal communication system, and benefits the end user by being cheap and easy to use.
	}
	\paragraph{}{
	The application will be developed as a Windows Universal App. With the release of Windows 10, Microsoft allowed developers to create applications that would run on all Windows 10 devices. While a similar concept was seen with Windows 8.1, these applications only require a single codebase. This provides the developer with the ability to deploy on multiple devices, such as phones, laptops and tablets, while also promoting maintainability of the code. It made sense to create an application for Windows, as it is the de facto standard operating system for home users, with all distributions of Windows holding a combined share of over 88{\%} of the operating system market share %\cite{OSMarketShare}.
	}
	\paragraph{}{
	As the application is Windows based, portability is an issue, as it would require a partial or full rewrite of the code to port it to Android or iOS, which is impossible given the timeframe of this project. Fortunately, the Xamarin framework allows for the creation of native applications for Android and iOS, using the C{\#} language. This means the code written for the existing application may be used for an Android port, with only a new UI needing to be created. This feature is of great importance to the project, not only to demonstrate the portability of the code, but to target more users.
	}
		
\section{Objectives}
	\paragraph{}{
	The key objectives that have been identified for this project are as follows:
		\begin{itemize}
			\item \textbf{Allow users to retrieve diagnostic information from their vehicle}\\
			The primary objective of this project is to create an application that will allow a user to communicate with and retrieve information from their vehicle. However, it is of paramount importance that the application is easy to use and comprehend without the user having any technical knowledge of their vehicle.
			
			\item \textbf{Develop a highly extensible and portable system}\\
			A key goal of the project is the design and implementation of an extensible and portable architecture. This is to allow	the application to be easily expanded in future and to be deployed on numerous platforms to target a larger user base.

			\item \textbf{Ensure the system cannot cause damage to a vehicle}\\
			As the system is designed for use by non-technical stakeholders, there is a risk that they may misuse the application due to a lack of technical knowledge. The system must account for this during the design and implementation stages.
		\end{itemize}
	}
	
	%\paragraph{}{
	%The primary objective of this project is to create an application that will allow a user to communicate with and diagnose their vehicle. However, it is of paramount importance that the application is easy to use and comprehend without the user having any technical knowledge of their vehicle.  
	%}
	%\paragraph{}{
	%The most challenging aspect will be finding a balance between the correct level of abstraction for an average user to understand, whilst also providing enough detail that they have all the information they need to perform a diagnosis. There are existing applications that provide similar services, but they are more directed at mechanics and repair technicians, who have an in depth understanding of the technical aspects of vehicles.
	%}
	%\paragraph{}{
	%The secondary objective will be to gain experience working with new technologies such as the design and development of Windows 10 Universal Apps. This will involved learning how to use C{\#} and XAML for development and researching the UI design guidelines for Windows Universal Apps to ensure the application has the aesthetics and usability that it is expected of modern applications.
	%}
	
\section{Methodology}
	\paragraph{}{
	This project will follow a Scrum development process, operating on week long sprints, including a retrospective with the project supervisor at the end of the week.
	}
	\paragraph{}{
	Planning will take place at the start of the project. During this phase, a project roadmap will be created, outlining the tasks and deliverables for each sprint. However, as requirements change and new information comes to light, it will be important to update the project plan over the course of the development cycle. This will occur during the retrospective, if necessary.
	}
	\paragraph{}{
	The basic system architecture will be designed at the start of the project. The intention is to create an extensible, modular architecture and the lightweight upfront design will provide a template to follow throughout the development cycle. After the basic architecture has been designed, all further design work will be done on a per module basis. This includes system and UI design work.
	}
	\paragraph{}{
 	The approach to implementation will be similar to that of design. A basic proof of concept and some prototypes will be created at the start of the project, before moving to an iterative approach. There will be a focus on writing code that has high readability and reusability. The code will be stored in a source control repository, to ensure that a backup is always available, and to keep track of code changes. Testing is an important aspect in the project, so unit tests will be written that will run before committing code to the source control repository. The application will undergo field testing as mentioned above. A variety of vehicle makes and models will be tested to find any defects and ensure the application functions as expected. 
	}

\section{Motivation}
	\paragraph{}{
	In recent years, it has become more commonplace for end users to self diagnose issues with their personal belongings. If an end user has an issue with a laptop or phone, they will usually search the internet for more information about the severity of the issue, rather than contacting the customer care service directly. This project intends to allow end users to apply that process to issues with their cars.
	}
	\paragraph{}{
	If a vehicle is taken to a mechanic for diagnosis or repair, they will usually connect to the car using a commercial scan tool. This will gather data from the vehicle in order to help diagnose the problems with the vehicle. The application will allow the end user to perform the same steps the mechanic performs with the commercial scan tool and provide them with the appropriate information before they go to the mechanic.
	}
	\paragraph{}{
	The average vehicle owner may not have technical knowledge of the inner workings of their vehicle. This makes it very difficult to estimate the cost of repairs. In order to estimate the cost of repair, one first needs to find the problematic parts of the vehicle. Once these parts have been identified, these parts can be sourced online. However, without technical knowledge or guidance from someone with it, the parts cannot be identified, let alone categorized into functional and faulty. By providing a list of possible causes and parts that may need to be replaced, the application gives the end user a starting point from which they can estimate the total cost of repairs. 
	}

\section{Overview of Report}
	\paragraph{}{
	Chapter 1 of the report gives a brief summary of the project and outlines the objectives, motivation and methodology adopted for this project. Chapter 2 will discuss the background research done for this project. This will include existing knowledge of the domain as well as information on vehicle communication, Windows 10 development and the Xamarin framework. Chapter 3 will outline the functional and non-functional requirements and how they will be met. Chapter 4 will outline and discuss the design decisions made for the system and user interface. Chapter 5 will summarise the implementation and testing processes and practises adopted during the development cycle, highlighting any issues that occurred and how they were dealt with.
	}	
	%\paragraph{}{
	%In this report, I will discuss my findings from my research into vehicle communication, Windows 10 development and the Xamarin framework, as well my existing knowledge of these areas. A brief description of the functional and non-functional requirements will be given, along with details regarding the tactics used to support them. 
	%}
	%\paragraph{}{
	%There will be an outline of the overall system design and user interface design. These will include diagrams and mock-ups, as well as a discussion on my motivation behind my design choices. This will be followed by the implementation and testing process adopted for this project, which will include the project plan and information on problems that occurred and how they were resolved. Finally, I will evaluate the end product, critique my development process and discuss future plans for the project.
	%}