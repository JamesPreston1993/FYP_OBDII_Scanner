	\paragraph{}{
	The objectives of this project were outlined in chapter 1. To determine if the project was a success, it must be evaluated to ensure the objectives were completed. 
		\begin{itemize}
			\item \textbf{Allow users to retrieve diagnostic information from their vehicle}\\
			The system includes the ability to connect to a vehicle, establish communication and retrieve diagnostic information. This information is displayed in a format that a non-technical end user can comprehend.
			
			\item \textbf{Develop a highly extensible and portable system}\\
			The system is highly extensible as it is easy to add new modules during development and required little to no refactoring to implement new features. The system is portable, as it is able to run on both Android and Windows 10 using the same core code. 

			\item \textbf{Ensure the system cannot cause damage to a vehicle}\\
			The system has been developed in a secure manner, so that the end user has no direct line of communication with the vehicle. The application ensures that accidental misuse can not lead to vehicle damage. 
		\end{itemize}
	}
	\paragraph{}{
	The key objectives of the project were achieved, as shown above. The project was also entered in a competition for "Most Commercially Viable FYP". The project came in third place overall, further highlighting the success of this project as a potential commercial venture.
	%Objectives were achieved, Won an award
	}
	
	\paragraph{}{
	The implementation of this project is based on:
	\begin{enumerate}
		\item Extensive study of the literature to identify and utilise knowledge necessary for a successful outcome
		\item Sound engineering principles for requirements analysis, architecture specification, and design, leveraging methods, techniques and tools appropriate to the context
	\end{enumerate}
	
	The subsequent codebase exhibits very good support for the quality attributes of interest: extensibility, security, and portability. The product's usability was evaluated in live field studies and found to be highly usable through quantitative research. All aspects of development were subjected to continuous evaluation in order to support continuous improvement. The overall development results from diligent and consistent application through both academic terms and the the intervening holiday period.
	}

	\paragraph{}{
	While the project has been a success, there are still a number of possibilities for future expansions of the project in the form of adding new functionality and porting the system to new platforms.
	}
			
	\paragraph{}{
	As the project implemented a subset of the modes shown in Table \ref{tab:Modes}, there is scope for extension of the application in this area. This will involve holding exploratory interviews with domain experts, such as mechanics, to determine what new features would complement the existing functionality of the application as well as user interface design for these new modules.
	%[ADD NEW MODES]
	}
	
	\paragraph{}{
	As a key aspect of the project was the development of a portable system, there is scope for this application to be ported to a number of new platforms. As Xamarin supports porting to iOS, Mac and older Windows platforms, a decision will be made as to which platform will be used. However, it is likely that iOS will be the next platform for the system, as it would make sense to target all mobile platforms first.
	%[PORT TO iOS]
	}
	\paragraph{}{
	In conclusion, the system has achieved the goals outlined at the beginning of the project and there is both scope and motivation to expand the system in future, through the implementation of new features and deployment on additional platforms.
	}