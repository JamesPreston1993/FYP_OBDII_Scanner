\subsection{Introduction}
	\paragraph{}{
	}

\subsection{Functional Testing}
	\paragraph{}{
	}
	
\subsection{Non-functional Requirements}
	\paragraph{}{
	}
	
\subsection{Software Quality}
	\paragraph{}{
	}
		
\subsection{Usability Testing}
	\paragraph{}{
	}
	
	
	%\paragraph{}{
	%Testing of the application is essential in ensuring that the system satisfies the functional and non-functional requirements.
	%}
	%\paragraph{}{
	%Ensuring the application displays the correct data is paramount, as an erroneous data value could cause confusion for the end user. Functional testing will ensure that data is not incorrectly converted by the communication system or incorrectly displayed by the UI. The output of the application will be compared with the output of the similar applications outlined in section 2.4, to ensure there is no disparity. 
	%}
	%\paragraph{}{
	%Unit tests will be written for each component of the application. These unit tests will be run before a change is ready to be committed to the source control repository. If any unit tests fail, that have not failed prior to the implementation of the recent changes, the code will not be committed until the underlying causes have been fixed. This helps to minimise the number of new defects introduced as part of implementing new features.
	%}